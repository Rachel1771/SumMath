\documentclass[UTF8]{ctexart}


% \documentclass{article}
\usepackage{amsmath}
\usepackage{listings}
\usepackage{color,xcolor} 
\usepackage{colortbl}
\usepackage{graphicx}
\usepackage{booktabs} %绘制表格
\usepackage{caption2} %标题居中
\usepackage{geometry}
\usepackage{array}
\usepackage{amsmath}
\usepackage{subfigure} 
\usepackage{longtable}
\usepackage{abstract}
\usepackage{multirow}
\usepackage{enumerate}
\usepackage{float}
\usepackage{graphicx}

\usepackage{xltxtra}
\usepackage{mflogo,texnames}
\usepackage{amssymb}

%伪代码
\usepackage{algorithm}
\usepackage{algpseudocode}
\usepackage{amsmath}
\renewcommand{\algorithmicrequire}{\textbf{输入:}}  % Use Input in the format of Algorithm
\renewcommand{\algorithmicensure}{\textbf{过程:}} % Use Output in the format of Algorithm


%for long table
\usepackage{longtable}

%for table toprule line
\usepackage{booktabs}

%附录代码
\usepackage{listings}
\usepackage{xcolor}
\lstset{
    numbers=left, 
    numberstyle= \tiny, 
    keywordstyle= \color{ blue!70},
    commentstyle= \color{red!50!green!50!blue!50}, 
    frame=shadowbox, % 阴影效果
    rulesepcolor= \color{ red!20!green!20!blue!20} ,
    escapeinside=``, % 英文分号中可写入中文
    xleftmargin=2em,xrightmargin=2em, aboveskip=1em,
    framexleftmargin=2em
} 

\pagestyle{plain} %页眉消失

\geometry{a4paper,left=2.5cm,right=2.5cm,top=2.5cm,bottom=2.5cm}%设置页面尺寸
\lstset{
		numbers=left, %设置行号位置
		numberstyle=\tiny, %设置行号大小	
		keywordstyle=\color{blue}, %设置关键字颜色
		commentstyle=\color[cmyk]{1,0,1,0}, %设置注释颜色
		escapeinside=``, %逃逸字符(1左面的键),用于显示中文
		breaklines, %自动折行
		extendedchars=false, %解决代码跨页时,章节标题,页眉等汉字不显示的问题
		xleftmargin=1em,xrightmargin=1em, aboveskip=1em, %设置边距
		tabsize=4, %设置tab空格数
		showspaces=false %不显示空格
	}


		% \begin{figure}[!htbp]\centering
		% \includegraphics[width=1\textwidth]{img} % 图片相对位置
		% \label{fig:figure 0} % 图片标签
		% \end{figure}


\title{基于\textbf{XXXXX}XX的生产企业原材料的订购与运输方案}
\date{September 2, 2022}

\begin{document}
\maketitle{}
\renewcommand{\abstractname}{\Large 摘要\\}
\begin{abstract}
	\normalsize
	葡萄酒备受大部分人的热爱,质量较好的葡萄酒往往带来更好的感官体验。葡萄酒的评价有着不同的指标,但其好坏却主要来自于评酒师的个人主观评价,所以会导致评价结果的差异性和不够客观性,存在着不同大众的口味喜爱偏差。本文为了解决该问题,通过探索酿酒葡萄对葡萄酒质量的影响,通过建立模型的方式更合理的对酿酒葡萄质量分类,以及如何依赖不同的酿酒葡萄和葡萄酒之间的关系来客观的给出葡萄酒评价结果,通过客观修正的数据建立了数学模型,以客观的方式来给出葡萄酒的评价。

	针对问题一,由于数据较多,且分布不均的原因,首先通过对数据的筛选处理,发现两组对红葡萄酒和白葡萄酒的打分情况是两两相互比较和配对进行K-S检验判断数据是否符合正态分布,再对样本T检验进行显著性差异判断,通过方差齐性检验来假设检验,最后依据显著性差异水平来判断打分组别的稳定性和可靠性。

	针对问题二,需要对酿酒葡萄进行分类,需要考虑到所酿葡萄酒的好坏。由于评酒员的打分存在主观性的干扰,我们将打分数据重新加权处理求和,得到一份较为中肯的综合评价结果。对酿酒葡萄和葡萄酒理化指标进行标准化处理,去除机值,后利用主成分分析对酿酒葡萄和葡萄酒理化指标提取主成分。将酿酒葡萄的数据和葡萄酒理化指标数据进行关联,采用改进的K-means++分类模型,通过对主成分聚类分析将葡萄酒质量分为六大类,由于部分葡萄酒得分评价数值较低,以此为下限进行分类,借鉴罗伯特帕克葡萄酒评分体系依据酿酒葡萄和葡萄酒的相关性,以酿酒葡萄的理化指标含量和分布来进行排名,分出酿酒葡萄的等级和优劣。


	1. yhlleo

	2. \textbf{yhlleo}

	3. $\textbf{yhlleo}$

	4. $\mathbf{yhlleo}$

	5. $\theta_i$

	%   6. $\textbf{\theta}_i$

	7. $\mathbf{\theta}_i$

	8. $\pmb{\theta}_i$


	针对问题三,探求酿酒葡萄和葡萄酒的理化指标之间的关系。首先对数据进行筛选处理出关键数据,剔除部分多余数据。提取葡萄酒某指标以及对应的酿酒葡萄理化指标进行相关性分析,通过perason系数和显著性水平来判断其相关性的强弱。后获取到相关性较强的理化指标数据进行共线性诊断后,进行德宾沃森残差分析后通过多元回归的方法来进行拟合,以R方的值来判断拟合的契合度高低,以高的模型数据为准得到葡萄酒和对应酿酒葡萄理化指标之间的系数,建立其函数关系。

	针对问题四,借鉴问题二中使用到的主成分分析法先对酿酒葡萄的理化指标进行分类,提取到八大类主成分,分析各主成分中相关较高的酿酒葡萄和葡萄酒的理化指标,从而得到对葡萄酒质量贡献较大的因素指标。对评分数据进行加权求和、去极值处理后。以评分标准数据作为因变量,酿酒葡萄理化指标主成分作为自变量来进行回归分析,得出回归的评分表。通过误差分析来分析以理化指标得出的评价和已给出的评价之间的误差,来判断是否能以葡萄酒和酿酒葡萄的理化指标作为葡萄酒质量的评价标准。

	\textbf{关键字}:K-S检验  聚类分析  主成分分析 K-means++分类模型 相关性分析 多元回归	Perason系数

\end{abstract}


\section{问题背景与重述}
\section{问题背景}
葡萄酒是当今世界上最畅销的酒类之一,在各种场合都有葡萄酒的身影。然而葡萄酒的酿造是取决于多种因素,各种因素的叠加会导致葡萄酒的品质差异明显。在不同的原材料已经酿制方法的差别下葡萄酒会继续细分,例如红葡萄酒和白葡萄酒等。对各种葡萄酒的鉴别是必不可少的一个步骤,而采用人工品尝打分和采用仪器进行理化指标的检验已成为最为科学的鉴别方法。最后经过安全检查、筛选分级的葡萄酒方可上市成为饮用酒。
\subsection{题目所给信息及参数}
此次比赛是根据10位品酒员为27款红酒和28款白葡萄酒的打分,以及上述葡萄酒的指标情况和芳香物质为基础进行数学分析和建模,并探讨品酒员的打分是否合理以及论证是否科学分级。红葡萄酒和白葡萄酒的市场在国际上的价值非常之高,葡萄酒依旧是未来的主力酒类,对此分析依旧存在价值。现在根据三个数据文件,并对三个数据进行分析处理后描述统计,完成数学建模和预测。

数据一:葡萄酒品尝评分表;数据二:指标总表;数据三:芳香物质。
\subsection{所需解决的问题}
1. 根据附件所提供的两组品酒员对27款红葡萄酒和28款白葡萄酒的打分判断两组结果是否有显著性差异,并判断哪一组的更加可信。

2. 根据酿酒葡萄的理化指标和葡萄酒的质量,使用无监督方法计算相似度,通过相似度进行分级。

3. 分析酿酒葡萄和葡萄酒的理化指标之间是否具有相关性,以及其之间具有什么样的联系。

4. 通过主成分分析对数据进行降维,得到贡献值大的特征进行表示,建立其函数关系,根据建立的函数进行预测,将预测结果与评分标准进行误差分析,判断建立的模型是否合理。



\section{问题一的分析}

第一,根据附件1中给出信息分析供货商的供货能力。对于此问题,分析附件一所提供的企业订货量和供应商供货数进行特征提取和量化分析,提取特征。

第二,通过PCA对建立的特征值进行主成分分析,通过主成分分析建立打分模型,计算不同供应商的分数,通过分数进行排序选择50家最重要的供应商。


\section{问题二的分析}
基于改进的K-means++\cite{arthur2006k}进行分类模型,为了降低数据数,首先对红、白葡萄和葡萄酒理化指标采用主成分分析法提取出主成分,但是经过Bartlett球形度检验\cite{arsham2011bartlett}等发现不适合进行主成分分析,进行标准化处理,去除极值,使评价分数更加客观,然后借鉴Robert Parker葡萄酒评分体系\cite{hommerberg2011persuasiveness},对这些主成分聚类分析得出6种聚类并依据判别标准(聚类后葡萄酒样本的平均值),最终确定红、白葡萄的分级。

\section{问题三的分析}
首先要参照附件所给的数据来进行分析酿酒葡萄和葡萄酒之间的相关性,附件的信息内容过多,需要进行合理的过滤和筛选数据,
但要尽可能保证其数据的完整性和真实性。我们采取相关性分析,依据相关性皮尔森系数来判断葡萄酒的数据和酿酒葡萄的理化指标之间的相关性显著程度。

在进行了相关性分析后,可以确定下一些具有显著相关的数据流,要进一步解决其葡萄酒某指标与该些理化指标的关系,
则要进行其关系的拟合,从而得出实质性的结论来判断酿酒葡萄和葡萄酒之间的关系,以及其关系的可靠性。
\section{问题四的分析}


\section{符号说明}
\begin{table}[!htbp]
	\begin{center}
		\begin{tabular}{c|c}
			\toprule[2pt]
			\rowcolor[gray]{0.8}

			\multicolumn{1}{m{8em}}{\centering 符号} & \multicolumn{1}{m{30em}}{\centering 基本说明}          \\

			%直接用合并单元格的方法来实现自定义列宽的同时,使文字居中对齐

			\midrule[1.3pt]
			$Goods$                                  & 订货量和生产产品                                       \\
			$W_i$                                    & 产品种类对应的每立方米产品需消耗对应种类的原材料       \\
			$Goods_{i_{std}}$                        & 第i个供货商(订货商)对应的第j周的供货量的标准差。     \\
			$Goods_{i_{mean}}$                       & 中第i个供货商(订货商)对应的第j周的供货量的均值。     \\
			$Goods\_{cv}_{i_{mean}}$                 & 即为第i个供货商(订货商)对应的第j周的变异系数         \\
			$Goods\_stab_{i}$                        & 为第i个供货商(订货商)对应的每一周的稳定指数          \\
			$Goods\_diff_{i}$                        & 为第i个供货商(订货商)对应的每一周的偏移指数          \\
			$diff\_metric$                           & 为订货的稳定指数                                       \\
			$vacancy\_rate$                          & 供货商占有率,$data\_orger_{ij}$为供货商i第j周的供货数 \\
			$time_{total}$                           & 为总周数                                               \\
			$Default\_rate$                          & 供应商违约率                                           \\
			$Compliance\_{rate}$                     & 供应商守约率                                           \\
			$important\_freq$                        & 重要订单接收频次                                       \\

			$segmentation_market_share$              & 为供应商i对应的市场份额                                \\

			\bottomrule[2pt]
		\end{tabular}
	\end{center}
\end{table}
\section{模型假设}

% 模型假设部分
\begin{itemize}
	\item [\bf{1)}]\bf{考虑到目前已处于疫情控制阶段,且人们自我隔离意识较好,不妨设每个病人的有效日接触率为定值}
	\item [2)]\bf{所有人口都为易感染者,不考虑个别免疫体质}
	\item [3)]\bf{疑似病例一旦确诊即被隔离,不会再传播给他人}
	\item [4)]\bf{隔离人群中一旦被确定未感染,则立刻结束隔离,即恢复易感者身份}
	\item [5)]\bf{在考虑隔离者与未隔离者时将确诊感染者和潜伏期患者都定义为感染者}

\end{itemize}


\section{问题一的求解}
\subsection{基于综合主成分分析评价法和秩和比综合评价法的供货能力评价模型的建立}

要求建立一套能够反映保障企业生产重要性的数学模型,在此基础上确定50家对于该建筑和装饰板材的生产企业的生产有重要性或者重要保障的原材料供应商。由于供应商对于企业的生产的影响是多方面多层次的,为此本文通过建立关于供应商供货能力的评价模型,给附件一中402家供应商综合打分,由该得分反映出供货商对于保障企业生产的重要程度,最后挑选出排名前50家供应商,确定为对该生产企业最重要的50家供应商。

\subsection{数据处理}
我们对于附件 1 的数据,运用 Python 的 pandas 库,找出缺失值。若有缺失值,可删除有缺失值的样本或特征,或对缺失值进行填补。检查数据是否存在缺失值或者NAN 现象,通过简单的查看可知该题数据不存在缺失,无需进行缺失值处理。

\subsection{评价指标的选取}
参考国内外已有研究用,发现在现有研究中“质量、成本、交货、服务"四个方面是人们评价供应商时关注的重点,通过分析供应商的供应特征以及选择供应商的主要目的,我们深度挖掘附件-中402家原材料供应商在近五年内的订货量和供货量的数据,从供货水平、订单完成水平和原材料供应类别这三方面进行细化指标选取,从而选取每周最大供货量、每周平均供货量、近5年总供货量、近5年订货频次、供货量的标准差、平均综合偏差值、综合偏差值的标准差、单位产品消耗A、B、C三类原材料的综合成本(分为采购和运输及储存两方面)八个二级的细化指标用以评价供应商的供货能力,具体的细化指标的计算公式如下。

\subsection{评价指标的量化计算}

\subsubsection{加权处理}
由于生产每立方米的材料需要消耗不同立方米的原材料,为了衡量不同原材料的重要性,我们在原材料数上除以产品种类对应的每立方米产品需消耗对应种类的原材料,统一三种材料的量纲。

\begin{equation}
	Goods\_wight_{ij} = \frac{Goods_{ij}}{W_{ij}}
\end{equation}

\subsubsection{供货订货差异性分析}
为了衡量供货与订货之间的稳定性,本文定义了供货订货比,用于衡量供货商的供货水平以及供货稳定性。对于供货订货比:等于1,说明供货与订货相等。小于1,说明供货不足,供货商私信供货能力不足。大于1,说明供货商提供的货物大于订货,可能会影响到后续的需求。
计算各订货量$D_{i}$ 与供货商生产的产品数$S_i$的比率$R_{i}$:

\begin{equation}
	R_{i} = \frac{D_{i}}{S_{i}}
\end{equation}

\subsubsection{供货商稳定性分析}
为了比较供货与订货两个数据离散程度,消除测量尺度和量纲的影响,导致两组数据因统计导致尺度相差太大,或者数据量纲的不同,直接使用标准差来进行比较不合适,,而变异系数可以做到这一点,它是原始数据标准差与原始数据平均数的比。CV没有量纲,这样就可以进行客观比较了。
$Goods\_{cv}_{i_{mean}}$即为第i个供货商(订货商)对应的第j周的变异系数,公式定义如下:

\begin{equation}
	Goods\_{cv}_{i_{mean}} =\frac{Goods_{i_{std}}}{Goods_{i_{mean}}}
\end{equation}

$Goods\_stab_{i}$为第i个供货商(订货商)对应的每一周的稳定指数:

\begin{equation}
	Goods\_stab_{i} = \frac{1}{Goods\_{cv}_{i_{mean}}+1}
\end{equation}

\subsubsection{供货偏移指数}
为衡量供货商是否能及时供货,我们提出使用供货偏移指数进行建模,判断供货商对于订货商给出订单需求后是否及时供货

$Goods\_diff_{i}$为第i个供货商(订货商)对应的每一周的偏移指数:
\begin{equation}
	Goods\_diff = Goods\_stab\_sup_{i} - Goods\_stab\_ord_{i}
\end{equation}

$diff\_metric$为订货的稳定指数,

\begin{equation}
	diff\_metric =\frac{1}{\frac{\sum^{i}_{n=0}(\frac{Goods\_diff_{n}}{Goods\_diff_{n}})^2}{i}+1}
\end{equation}

\subsubsection{供货商实力评判}
为了对比供货商之间的实力,我们通过计算供货商提供的总货物数占对应种类市场的份额进行评判比较判断,所占市场份额越大,则说明对应供货商的能力越强。

\begin{equation}
	order\_bool_{ij}=\left\{\begin{array}{llcl}

		True  & data\_order_{ij}=0    \\
		False & data\_order_{ij}\neq0
	\end{array} \right.
\end{equation}

\begin{equation}
	data\_order\_IdleI_i=\frac{\sum_{n=0}^j order\_bool_{in}}{time_{total}}
\end{equation}

\begin{equation}
	vacancy\_rate_i = 1-data\_order\_Idle_i
\end{equation}


\subsubsection{供货商重要订单接收频次}

$important\_freq$被初始化为全0的矩阵。
\begin{equation}
	max\_order_{i} =\{data\_order_{ij}|data\_order_{ij}∈argsort(data\_order_{ij}) \wedge 0\leq j\leq20\}
\end{equation}

\begin{equation}
	important\_freq_{i} =important\_freq_{ij} +1
	\\
	s.t.\ max\_order_{ij} ∈ max\_order_{i}
\end{equation}

\subsubsection{供货商信誉判断}
为了判断供货商供货,我们通过计算订货商订货,但是供货商没有供货的天数来判断。

\begin{equation}
	data\_order\_bool_{ij} =\left\{\begin{array}{llcl}

		True  & data\_order_{ij}>0     \\
		False & data\_order_{ij}\leq 0
	\end{array} \right.\\
\end{equation}

\begin{equation}
	data\_supply\_bool_{ij} =\left\{\begin{array}{llcl}

		True  & data\_supply_{ij}=0     \\
		False & data\_supply_{ij}\neq 0
	\end{array} \right.
\end{equation}

\begin{equation}
	data\_bool2_{ij} = =\left\{\begin{array}{llcl}

		True  & data\_order\_bool_{ij}\&data\_supply\_bool_{ij}==True  \\
		False & data\_order\_bool_{ij}\&data\_supply\_bool_{ij}==False
	\end{array} \right.
\end{equation}


\begin{equation}
	data\_order2\_bool_{ij} =\left\{\begin{array}{llcl}

		True  & data\_order_{ij}\neq0 \\
		False & data\_order_{ij}= 0
	\end{array} \right.
\end{equation}

\begin{equation}
	Default\_rate_i = \frac{\sum_{n=0}^i{data\_bool_{in}}}{\sum^i_{m=0}data\_order2\_bool_{im}}
\end{equation}

\begin{equation}
	Compliance\_{rate} = 1- Default\_rate
\end{equation}


\subsection{基于综合主成分分析评价法和秩和比综合评价法的供货能力评价模型}

\subsubsection{分析步骤}

1. 准备好数据,并且进行同趋势化处理与量纲问题;

2. 确认各指标权重,可使用熵权法、自定义权重、层次分析法(需自行处理,可使用SPSSPRO量化分析-AHP);

3. 计算秩值,根据每一个具体的评价指标按其指标值的大小进行排序,得到秩次R,用秩次R来代替原来的评价指标值;

4. 计算得到RSR值和RSR值排名;

5. 列出RSR的分布表格情况并且得到Probit值;

6. 以Probit值(累积频率所对应的概率单位)为自变量,以 RSR 值为因变量,计算直线回归方程,拟合所对应的RSR估计值;

7. 根据拟合的RSR值排序,并且进行分档等级。

\subsubsection{秩和比综合评价法(RSR)分析结果}

使用熵权法进行指标权重计算,下表展示了熵权法的权重计算结果,根据结果对各个指标的权重进行分析。熵权法的权重计算结果显示,订货数量的权重为21.228\%、重要订单接受频次的权重为23.209\%、供货数量的权重为23.762\%、供应商细分市场份额的权重为23.138\%、订货稳定指数的权重为2.216\%、供应商占用率的权重为2.765\%、供应商守约率的权重为1.405\%、供货稳定指数的权重为1.158\%、供货偏移指数的权重为1.118\%,其中指标权重最大值为供货数量(23.762\%),最小值为供货偏移指数(1.118\%)。

\begin{table}[!ht]
	\centering
	\caption{熵权法}
	\begin{tabular}{|c|c|c|c|}
		\hline
		项                 & 信息熵值e & 信息效用值d & 权重   \\ \hline
		订货数量           & 0.643     & 0.357       & 21.228 \\ \hline
		重要订单接受频次   & 0.61      & 0.39        & 23.209 \\ \hline
		供货数量           & 0.6       & 0.4         & 23.762 \\ \hline
		供应商细分市场份额 & 0.611     & 0.389       & 23.138 \\ \hline
		订货稳定指数       & 0.963     & 0.037       & 2.216  \\ \hline
		供应商占用率       & 0.953     & 0.047       & 2.765  \\ \hline
		供应商守约率       & 0.976     & 0.024       & 1.405  \\ \hline
		供货稳定指数       & 0.981     & 0.019       & 1.158  \\ \hline
		供货偏移指数       & 0.981     & 0.019       & 1.118  \\ \hline
	\end{tabular}
\end{table}

进行秩值计算,根据每一个具体的评价指标按其指标值的大小进行排序,得到秩次R,用秩次R来代替原来的评价指标值,根据编秩结果建立各指标的秩次数据矩阵。使用非整秩法进行计算,类似于线性插值的方式对指标值进行编秩,以改进 RSR 法编秩方法的不足,所编秩次与原指标值之间存在定量的线性对应关系,从而克服了 RSR 法秩次化时易损失原指标值定量信息的缺点。下表展示部分秩值计算结果。


TODO::这种太宽的表格怎么办(T)
\begin{center}
\begin{longtable}{|c|c|c|c|c|c|c|c|c|c|c|c|c|c|c|c|c|c|c|c|c|}
\caption{部分秩值计算结果}
\label{tab:dasfa}  \\
\hline
索引 & X1:订货数量 & R1:订货数量 & X2:重要订单接受频次 & R6:供应商占用率 & X7:供应商守约率 & R7:供应商守约率 & X8:供货稳定指数 & R8:供货稳定指数 & X9:供货偏移指数 & R9:供货偏移指数 & RSR         & RSR排名                                       \\ \hline
0    & 0.000536236  & 1.215030487  & 1.00E-04             & 0.178683732      & 72.65217652      & 0.352204757      & 142.2341077      & 0.252111567      & 102.0967382      & 0.285493134      & 115.4827467 & 0.912283347 & 366.825622  & 0.033442045 & 241 \\ \hline
1    & 0.000762555  & 1.305784611  & 1.00E-04             & 0.187645018      & 76.2456522       & 0.369592433      & 149.2065656      & 0.739424303      & 297.5091455      & 0.217836655      & 88.3524986  & 0.615109636 & 247.6589642 & 0.037200684 & 190 \\ \hline
2    & 0.027258321  & 11.93058687  & 0.029260815          & 0.375976723      & 151.766666       & 0.821671999      & 330.4904716      & 0.958448159      & 385.3377116      & 0.446281899      & 179.9590414 & 0.211511912 & 85.81627684 & 0.082917866 & 48  \\ \hline
3    & 0.001537944  & 1.616715407  & 1.00E-04             & 0.148061779      & 60.3727734       & 0.404367784      & 163.1514814      & 0.299191902      & 120.9759527      & 0.296433212      & 119.8697181 & 0.874121452 & 351.5227021 & 0.034795266 & 220 \\ \hline
4    & 0.015002396  & 7.015960762  & 1.00E-04             & 0.627125997      & 252.4775247      & 0.452183892      & 182.3257407      & 0.936587585      & 376.5716217      & 0.729115356      & 293.3752579 & 0.178008933 & 72.3815823  & 0.064748189 & 62  \\ \hline
5    & 0.000936295  & 1.375454444  & 1.00E-04             & 0.161435272      & 65.73554407      & 0.195715677      & 79.48198637      & 0.212559285      & 86.23627314      & 0.279476034      & 113.0698896 & 0.927562411 & 372.9525267 & 0.028351586 & 365 \\ \hline
6    & 0.015409314  & 7.179134844  & 0.008431647          & 0.371974055      & 150.1615962      & 0.999895674      & 401.9581653      & 0.999896896      & 401.9586554      & 0.496297479      & 200.0152891 & 0.388416949 & 156.7551967 & 0.076823692 & 51  \\ \hline
7    & 0.000235238  & 1.094330557  & 1.00E-04             & 0.378469585      & 152.7663034      & 0.200062596      & 81.22510085      & 0.245030266      & 99.2571367       & 0.207662593      & 84.27269998 & 0.875828334 & 352.2071619 & 0.032181221 & 282 \\ \hline
8    & 0.0015421    & 1.618382149  & 1.00E-04             & 0.060921645      & 25.4295798       & 0.304388649      & 123.0598484      & 0.213730918      & 86.70609812      & 0.259219233      & 104.9469123 & 0.895519991 & 360.1035162 & 0.028682874 & 360 \\ \hline
9    & 0.001253226  & 1.502543552  & 1.00E-04             & 0.208543322      & 84.62587205      & 0.339164001      & 137.0047642      & 0.34378222       & 138.8566702      & 0.230205957      & 93.31258859 & 0.836868683 & 336.5843417 & 0.033848774 & 234 \\ \hline
10   & 0.000216188  & 1.086691321  & 1.00E-04             & 0.438579861      & 176.8705241      & 0.14355265       & 58.5646126       & 0.736143156      & 296.1934055      & 0.43135083       & 173.971683  & 0.59588097  & 239.9482691 & 0.038345178 & 176 \\ \hline
11   & 0.000760269  & 1.304867903  & 1.00E-04             & 0.193280991      & 78.50567719      & 0.252225622      & 102.1424746      & 0.150807315      & 61.47373336      & 0.217218853      & 88.10475997 & 0.957047621 & 384.7760959 & 0.029322827 & 345 \\ \hline
12   & 0.000139986  & 1.056134377  & 1.00E-04             & 0.455421798      & 183.6241408      & 0.082695785      & 34.16100988      & 0.679919788      & 273.647835       & 0.428467709      & 172.8155511 & 0.693549245 & 279.1132474 & 0.037241406 & 189 \\ \hline
13   & 0.000129318  & 1.051856405  & 1.00E-04             & 0.725164321      & 291.7908929      & 0.074001947      & 30.67478092      & 0.579844643      & 233.5177019      & 0.483087677      & 194.7181583 & 0.701089975 & 282.1370801 & 0.042256686 & 136 \\ \hline
\end{longtable}
\end{center}

以下表格为部分结果,RSR 的分布是指用概率单位 Probit 表达的值特定的累计频率 。
其方法为:
step 1 将RSR值按照从小到大的顺序排列;
step 2 列出各组频数;
step 3 计算各组累计频数;
step 4 确定各组RSR的秩次R及平均秩次 R-;
step 5 计算向下累计频率 $R- / n × 100 \%$, 最后一项用$( 1 − \frac{1}{4n}) × 100 \%$ 修正;
step 6 根据累计频率,查询“百分数与概率单位对照表”,求其所对应概率单位 Probit 值;
step 7 利用表格中的RSR分布值作为自变量,Probit值作为因变量,进行线性回归,结果如下表格。
PS:
● 在编秩过程中进行的是同向趋势化处理,即将负向指标(成本型指标)转化成正向指标(效益型指标),统一对所有指标进行从小到大编秩;

\begin{center}
\begin{longtable}{|c|c|c|c|c|c|}
	\caption{部分秩值计算结果}
		\label{tab:dasfa}  \\
		\hline
		RSR         & 频率f & 累计频数Σf & 评价秩数 & 评价秩数/n*100\% & Probit      \\ \hline
		0.020844231 & 1     & 1          & 1        & 0.248756219      & 2.191359989 \\ \hline
		0.024206993 & 1     & 2          & 2        & 0.497512438      & 2.422446536 \\ \hline
		0.024803811 & 1     & 3          & 3        & 0.746268657      & 2.565815209 \\ \hline
		0.024872558 & 1     & 4          & 4        & 0.995024876      & 2.671781372 \\ \hline
		0.025293148 & 1     & 5          & 5        & 1.243781095      & 2.756671237 \\ \hline
		0.025377881 & 1     & 6          & 6        & 1.492537313      & 2.827934812 \\ \hline
		0.025529463 & 1     & 7          & 7        & 1.741293532      & 2.889622706 \\ \hline
		0.025803867 & 1     & 8          & 8        & 1.990049751      & 2.944191672 \\ \hline
		0.025924679 & 1     & 9          & 9        & 2.23880597       & 2.993248406 \\ \hline
		0.026052561 & 1     & 10         & 10       & 2.487562189      & 3.037903443 \\ \hline
		0.026059396 & 1     & 11         & 11       & 2.736318408      & 3.078957646 \\ \hline
		0.026165731 & 1     & 12         & 12       & 2.985074627      & 3.117008298 \\ \hline
		0.026277061 & 1     & 13         & 13       & 3.233830846      & 3.152513182 \\ \hline
		0.026356117 & 1     & 14         & 14       & 3.482587065      & 3.185831195 \\ \hline
		0.02639066  & 1     & 15         & 15       & 3.731343284      & 3.217249125 \\ \hline
		0.026417487 & 1     & 16         & 16       & 3.980099502      & 3.246999901 \\ \hline
		0.026651563 & 1     & 17         & 17       & 4.228855721      & 3.275275394 \\ \hline
		0.026657858 & 1     & 18         & 18       & 4.47761194       & 3.302235607 \\ \hline
		0.026673477 & 1     & 19         & 19       & 4.726368159      & 3.328015422 \\ \hline
		0.026962146 & 1     & 20         & 20       & 4.975124378      & 3.352729641 \\ \hline
		0.027170188 & 1     & 21         & 21       & 5.223880597      & 3.376476818 \\ \hline
		0.02722017  & 1     & 22         & 22       & 5.472636816      & 3.399342216 \\ \hline
		0.027249725 & 1     & 23         & 23       & 5.721393035      & 3.421400113 \\ \hline
		0.027298297 & 1     & 24         & 24       & 5.970149254      & 3.442715632 \\ \hline
		0.027316883 & 1     & 25         & 25       & 6.218905473      & 3.463346196 \\ \hline
		0.027355379 & 1     & 26         & 26       & 6.467661692      & 3.483342714 \\ \hline
		0.027437135 & 1     & 27         & 27       & 6.71641791       & 3.502750533 \\ \hline
		0.027514565 & 1     & 28         & 28       & 6.965174129      & 3.521610232 \\ \hline
		0.027730933 & 1     & 29         & 29       & 7.213930348      & 3.539958275 \\ \hline
		0.027753687 & 1     & 30         & 30       & 7.462686567      & 3.557827553 \\ \hline
\end{longtable}
\end{center}

下表格展示了本次模型的分析结果,包括模型的标准化系数、t值、VIF值、$R^2$、调整$R^2$等,用于模型的检验,并分析模型的公式。
step 1 线性回归模型要求总体回归系数不为0,即变量之间存在回归关系。根据F检验结果对模型进行检验;
step 2 $R^2$代表曲线回归的拟合程度,越接近1效果越好;
step 3 VIF值代表多重共线性严重程度,用于检验模型是否呈现共线性,即解释变量间存在高度相关的关系(VIF应小于10或者5,严格为5);
若VIF出现inf,则说明VIF值无穷大,建议检查共线性,或者使用岭回归。

从F检验的结果分析可以得到,显著性P值为1\%,水平呈现显著性,拒绝了回归系数为0的原假设,同时模型的拟合优度$R^2$为0.38,模型表现较差,因此模型基本满足要求。对于变量共线性表现,VIF全部小于10,因此模型没有多重共线性问题,模型构建良好。对于变量共线性表现,VIF全部小于10,因此模型没有多重共线性问题,模型构建良好。
模型的公式如下:
\begin{equation}
	y=-0.296+0.073\cdot Probit
\end{equation}

\begin{table*}[!ht]
	\centering
	\caption{线性回归分析结果n=402}
	\begin{tabular}{|c|c c|c|c|c|c|c|c|c|}
		\hline
		~      & \multicolumn{2}{|c|}{非标准化系数} & 标准化系数 & \multirow{2}{*}{t} & \multirow{2}{*}{p} & \multirow{2}{*}{VIF} & \multirow{2}{*}{R2} & \multirow{2}{*}{调整R2} & \multirow{2}{*}{F}                        \\
		~      & B                                  & 标准误     & Beta               & \multirow{2}{*}{}  & \multirow{2}{*}{}    & \multirow{2}{*}{}   & \multirow{2}{*}{}       & \multirow{2}{*}{}  & \multirow{2}{*}{}    \\ \hline
		常数   & -0.296                             & 0.024      & -                  & -12.402            & 0.000***             & -                   & 0.38                    & 0.378              & F=245.181 P=0.000*** \\ \hline
		Probit & 0.073                              & 0.005      & 0.616              & 15.658             & 0.000***             & 1                   & ~                       & ~                  & ~                    \\ \hline
	\end{tabular}
\end{table*}


TODO:这种注放在哪里
注:
1. ***、**、*分别代表1\%、5\%、10\%的显著性水平。
2. 因变量:RSR。

下图为拟合曲线与真实曲线的对比

\begin{figure}[H]\centering
	\includegraphics[width=0.45\textwidth]{img/fit.png} % 图片相对位置 
	\caption{拟合效果图} % 图片标题 
	\label{fig:figure 1} % 图片标签
\end{figure}

最后,我们得到分档排序临界值表格,尤其是Probit临界值对应的RSR临界值(拟合值)


我们根据综合主成分分析评价法和秩和比综合评价法分别得到供货商的评分PCA得分与RSR拟合值,我们将他们进行min-max处理,去除两个评分的量纲,将两个评价指标求和得到的综合得分作为最终的评价指标,取评分最高的Top50为最终选取的供应商。
\begin{center}
	\begin{longtable}{|c|c|c|c|c|c|c|}
		\caption{第二问预测结果}
		\label{tab:dasfa}                                                                              \\
		\hline
		供货商 & Probit      & RSR拟合值   & PCA得分     & PCA(标准化) & RSR拟合(标准化) & 综合评分值  \\ \hline
		228    & 8.228645179 & 0.306420133 & 2.782049173 & 1           & 1               & 2           \\ \hline
		360    & 7.808640011 & 0.275660741 & 2.601285883 & 0.950070052 & 0.930431451     & 1.880501503 \\ \hline
		274    & 7.243328763 & 0.23425975  & 2.3525052   & 0.881352524 & 0.836794786     & 1.718147309 \\ \hline
		328    & 7.110377294 & 0.224522951 & 2.333973533 & 0.876233757 & 0.814773057     & 1.691006814 \\ \hline
		339    & 7.172065188 & 0.229040709 & 2.278486776 & 0.860907354 & 0.824990875     & 1.68589823  \\ \hline
		267    & 6.921042354 & 0.210656865 & 2.278209629 & 0.860830802 & 0.783412116     & 1.644242918 \\ \hline
		130    & 7.006751594 & 0.216933845 & 2.126081771 & 0.818810455 & 0.797608768     & 1.616419224 \\ \hline
		107    & 7.577553464 & 0.258736946 & 1.713112255 & 0.704741131 & 0.892154886     & 1.596896017 \\ \hline
		305    & 6.847486818 & 0.20526997  & 2.120079287 & 0.817152466 & 0.77122857      & 1.588381035 \\ \hline
		193    & 6.782750875 & 0.200528985 & 2.129456061 & 0.819742493 & 0.760505879     & 1.580248372 \\ \hline
		281    & 7.055808328 & 0.220526552 & 1.905487341 & 0.757878458 & 0.805734397     & 1.563612855 \\ \hline
		246    & 6.623523182 & 0.188867826 & 1.864908952 & 0.746670005 & 0.734131823     & 1.480801828 \\ \hline
		351    & 6.753000099 & 0.198350165 & 1.783104178 & 0.724074111 & 0.75557804      & 1.47965215  \\ \hline
		139    & 7.434184791 & 0.248237234 & 1.283004136 & 0.585937828 & 0.868407676     & 1.454345504 \\ \hline
		355    & 6.814168805 & 0.202829901 & 1.603297148 & 0.674408299 & 0.765709863     & 1.440118162 \\ \hline
		150    & 7.328218628 & 0.240476723 & 1.257439062 & 0.578876312 & 0.85085572      & 1.429732032 \\ \hline
		30     & 6.557284368 & 0.184016778 & 1.704744368 & 0.702429776 & 0.723160203     & 1.425589979 \\ \hline
		307    & 6.962096557 & 0.2136635   & 1.380909307 & 0.61298093  & 0.790212225     & 1.403193155 \\ \hline
		329    & 6.882991702 & 0.207870197 & 1.286324272 & 0.586854907 & 0.777109506     & 1.363964413 \\ \hline
		293    & 6.298749674 & 0.165082796 & 1.584520457 & 0.669221852 & 0.680337196     & 1.349559048 \\ \hline
		364    & 6.478389768 & 0.178238873 & 1.440791839 & 0.629521521 & 0.710092309     & 1.33961383  \\ \hline
		142    & 6.724724606 & 0.196279388 & 1.284431535 & 0.5863321   & 0.750894561     & 1.337226661 \\ \hline
		265    & 6.216273909 & 0.159042622 & 1.48804126  & 0.642572628 & 0.666676128     & 1.309248756 \\ \hline
		345    & 6.328298237 & 0.167246807 & 1.313626842 & 0.594396349 & 0.685231542     & 1.279627891 \\ \hline
		39     & 6.460041725 & 0.17689514  & 1.18975455  & 0.560180679 & 0.707053187     & 1.267233866 \\ \hline
		283    & 6.536653804 & 0.182505883 & 1.081315281 & 0.530227877 & 0.71974301      & 1.249970887 \\ \hline
		366    & 6.391152774 & 0.171850006 & 1.086761145 & 0.531732119 & 0.695642602     & 1.227374721 \\ \hline
		122    & 6.165587919 & 0.155330596 & 1.162793881 & 0.552733676 & 0.658280635     & 1.211014311 \\ \hline
		217    & 6.242830371 & 0.160987505 & 1.108545615 & 0.537749366 & 0.671074872     & 1.208824238 \\ \hline
		363    & 6.42475216  & 0.174310682 & 0.967951043 & 0.498914713 & 0.701207916     & 1.200122628 \\ \hline
		243    & 6.256443485 & 0.161984471 & 1.051499232 & 0.521992168 & 0.673329711     & 1.195321879 \\ \hline
		79     & 6.284391587 & 0.164031271 & 1.006025901 & 0.509431647 & 0.677958961     & 1.187390608 \\ \hline
		66     & 6.061292014 & 0.147692408 & 1.050982553 & 0.521849452 & 0.641005337     & 1.16285479  \\ \hline
		138    & 6.671984578 & 0.192416933 & 0.656463001 & 0.412876327 & 0.742158842     & 1.15503517  \\ \hline
		188    & 6.129453754 & 0.152684284 & 0.972738958 & 0.500237218 & 0.652295468     & 1.152532686 \\ \hline
		212    & 6.05040415  & 0.146895027 & 0.954365241 & 0.495162079 & 0.6392019       & 1.134363979 \\ \hline
		350    & 5.810051726 & 0.129292638 & 1.084013747 & 0.53097324  & 0.599390558     & 1.130363798 \\ \hline
		75     & 6.072307165 & 0.14849911  & 0.85290527  & 0.46713708  & 0.642829856     & 1.109966936 \\ \hline
		396    & 5.759152739 & 0.125565012 & 1.029222365 & 0.515838912 & 0.590959782     & 1.106798694 \\ \hline
		4      & 6.028993803 & 0.145327024 & 0.866036047 & 0.470764028 & 0.635655546     & 1.106419574 \\ \hline
		394    & 6.600657784 & 0.187193262 & 0.488672451 & 0.366529674 & 0.73034446      & 1.096874135 \\ \hline
		179    & 5.734430232 & 0.123754441 & 0.981421551 & 0.5026355   & 0.586864812     & 1.089500312 \\ \hline
		200    & 6.697764393 & 0.194304937 & 0.40293102  & 0.342846408 & 0.746428943     & 1.089275351 \\ \hline
		347    & 6.647270359 & 0.190606969 & 0.388700859 & 0.338915791 & 0.738065245     & 1.076981036 \\ \hline
		54     & 6.374926512 & 0.170661664 & 0.537650657 & 0.380058302 & 0.692954928     & 1.07301323  \\ \hline
		45     & 5.987529589 & 0.142290361 & 0.759295529 & 0.44128045  & 0.628787522     & 1.070067972 \\ \hline
		6      & 6.153375681 & 0.154436224 & 0.627358189 & 0.404837074 & 0.656257833     & 1.061094907 \\ \hline
		365    & 5.726288885 & 0.123158204 & 0.873465466 & 0.472816162 & 0.585516302     & 1.058332464 \\ \hline
		97     & 6.039639404 & 0.146106663 & 0.681039986 & 0.419664916 & 0.637418856     & 1.057083772 \\ \hline
	\end{longtable}
\end{center}



\subsubsection{各葡萄酒样本评分数据概率分布的确定}
对两组品酒员差异性评价的假设检验一般要求数据符合正态分布,因为两配对样本T检验的前提要求为数据符合正态分布,才可以使用T检验的数学模型。
利用 SPSS 统计软件中单样本 K-S 检验\cite{young1977proof}, 对数据集两组品酒员分别对红、白葡萄酒品尝得到的四组评价结果进行了正态分布检验。

% \begin{figure}[H]\centering
% 	\includegraphics[width=0.45\textwidth]{img/1/red_k_S.png} % 图片相对位置 
% 	\caption{聚类汇总图} % 图片标题 
% 	\label{fig:figure 1} % 图片标签
% \end{figure}

% \begin{figure}[H]\centering
% 	\includegraphics[width=0.45\textwidth]{img/1/white_k_S.png} % 图片相对位置 
% 	\caption{聚类汇总图} % 图片标题 
% 	\label{fig:figure 2} % 图片标签
% \end{figure}

从图1\label{figure 1}和图2\label{figure 2}可以看出两组的双边检验结果。因此可以认为品酒员对葡萄酒的评分服从正态分布。

\subsubsection{两组评价结果的显著性差异评价}
上述检验显示各类葡萄酒得分情况属于正态总体,为了进一步说明品酒员评分的科学性以及两个评分组评分的可信度, 需要检查两组给出的评分是否有显著性差异, 即对数据进行显著性检验。

两配对样本非参数检验一般用于同一研究对象分别给予两种不同处理的效果比较。因为两组品酒员分别对同一样本组进行评分,故两组数据为配对数据。

\begin{equation}
	z_{li} = w_{li}-w_{2i}(i=1,2,,,,n)
\end{equation}

$z_li$来自正态分布,用假设检验的方法,假设$H_{0}:u_1=0$成立;
\[\left\{\begin{array}{llcl}

		\bar{z}=\frac{1}{n}\sum_{i=1}^n(Z_li)         \\

		s_{1}^2=\frac{1}{n-1}\sum_{i=1}^n(z_li-z_1)^2 \\

		t=\frac{\bar{z_1}}{\frac{s_1}{\sqrt{n}}}      \\

		w={\mid t \mid \ge t_{1-\frac{a}{2}}(n-1)}    \\

		% 	% S(t_0)&=&140005\times10^4&\text{人},\\
		% 	% I(t_0)&=&14052 &\text{人},\\
		% 	% R(t_0)&=&130 &\text{人},\\
		% 	% Y(t_0)&=&4562 &\text{人},\\
		% 	% N(t_0)&=&6336 &\text{人},\\
	\end{array} \right.\]

对于统计量t,在给定显著性水平a下,该检验问题的拒绝域是w,若${\mid t \mid \ge t_{1-\frac{a}{2}}(n-1)}$,则拒绝假设反之则接受。

\begin{center}
	\begin{tabular}{||c c c c c c||}
		\hline
		组数 & 样本数 & 平均值 & 标准差 & $T_1$值 & p      \\ [0.5ex]
		\hline
		一   & 27     & 73.056 & 3.9780 & 2.458   & 0.0104 \\
		\hline
		二   & 27     & 70.515 & 7.3426 & 2.458   & 0.0104 \\
		\hline
	\end{tabular}
\end{center}

上表给出了两组红葡萄酒评分均值的t检验结果,通过查表当x =0.05, n=27时, $t_{1-\frac{a}{2}}(n-1)=2.0555<2.491$
且方差齐性检验的$p$值为0.0104<0.05,
所以拒绝原假设,对于红葡萄酒的评价,两组评酒员的评价结果有显著性差异。
因为第二组评酒员对红葡萄酒样品评分的标准差大于第一组的,第二组各评酒员得评分差异小,稳定性高,比较可信。

对于白葡萄酒采用同样的方法,得到了如下的表格:
\begin{center}
	\begin{tabular}{||c c c c c c||}
		\hline
		组数 & 样本数 & 平均值 & 标准差 & $T_1$值 & p       \\ [0.5ex]
		\hline
		一   & 28     & 74.261 & 5.2012 & -2.184  & 0.01892 \\
		\hline
		二   & 28     & 76.532 & 3.1709 & -2.184  & 0.01892 \\
		\hline
	\end{tabular}
\end{center}

上表给出了两组红葡萄酒评分均值的$t$检验结果,通过查表当$x=0.05$, $n=28$时, $t_{1-\frac{a}{2}}(n-1)=2.0555<2.491$且方差齐性检验的$p$值为0. 01892<0.05,所以拒绝原假设,对于白葡萄酒的评价,两组评酒员的评价结果有显著性差异。因为第一组评酒员对红葡萄酒样品评分的标准差大于第二组,第二组各评酒员得评分差异小,稳定性高,比较可信。

综上分别对两组葡萄酒进行T检验\cite{de2013using},在显著性水平为0.05时,得出两组评酒员的评价结果有显著性差异,第二组评酒员的评分更可信。


\section{问题二的求解}

\section{问题三的求解}


%引用
\clearpage
\bibliographystyle{plain}
\bibliography{ref}%ref指向自己创建的ref.bib
% IoU\cite{zheng2020distance}
\clearpage

\section{附录}
\subsection{代码}

\lstset{language=python}
\begin{lstlisting}
	import numpy as np
	import pandas as pd
	from copy import deepcopy
	
	# 数据读取
	data1 = np.array(pd.read_excel('./excel/附件1近5年402家供应商的相关数据.xlsx', sheet_name=0).fillna(0))
	data2 = np.array(pd.read_excel('./excel/附件1近5年402家供应商的相关数据.xlsx', sheet_name=4).fillna(0))
	
	data_order = data1[:, 2:]
	data_supply = data2[:, 2:]
	data_supplier = data1[:, 0:2]
	
	#  计算各供应商订货量/供货量能够生产的产品数
	data_order2 = np.zeros(data_order.shape)
	for i in range(data_order.shape[0]):
		if data_supplier[i, 1] == 'A':
			data_order2[i] = data_order[i] * (1 / 0.6)
		elif data_supplier[i, 1] == 'B':
			data_order2[i] = data_order[i] * (1 / 0.66)
		else:
			data_order2[i] = data_order[i] * (1 / 0.72)
	# data_order2= data_order2.sum()  / 240
	
	data_supply2 = np.zeros(data_supply.shape)
	for i in range(data_supply.shape[0]):
		if data_supplier[i, 1] == 'A':
			data_supply2[i] = data_supply[i] * (1 / 0.6)
		elif data_supplier[i, 1] == 'B':
			data_supply2[i] = data_supply[i] * (1 / 0.66)
		else:
			data_supply2[i] = data_supply[i] * (1 / 0.72)
	
	# 供货数量
	supply_num = data_supply2.sum(axis=1)
	# 供货稳定指数 1/(供货量变异系数+1)
	supply_stability = np.array(
		[np.std(data_supply2[k][data_order2[k] != 0]) / np.mean(data_supply2[k][data_order2[k] != 0])
		 for k in range(402)])
	supply_stability2 = 1 / (supply_stability + 1)
	# 供货偏移指数 1/(供货量与订货量差的平方平均数+1)
	diff = data_supply2 - data_order2
	supply_shift = []
	for k in range(402):
		shift_i = np.mean(np.square(diff[k][data_order2[k] != 0] / data_order2[k][data_order2[k] != 0]))
		supply_shift.append(shift_i)
	supply_shift = np.array(supply_shift)
	supply_shift2 = 1 / (supply_shift + 1)
	# 订货数量
	order_num = data_order2.sum(axis=1)
	# 订货稳定指数  1/(订货量的变异系数+1)
	order_stability = np.array(
		[np.std(data_order2[k][data_order2[k] != 0]) / np.mean(data_order2[k][data_order2[k] != 0])
		 for k in range(402)])
	order_stability2 = 1 / (order_stability + 1)
	# 供应商占用率 1-闲置率
	vacancy_rate = (data_order == 0).sum(axis=1) / 240
	vacancy_rate2 = 1 - vacancy_rate
	# 供应商守约率 1-违约
	default_rate = ((data_order > 0) & (data_supply == 0)).sum(axis=1) / (data_order != 0).sum(axis=1)
	default_rate2 = 1 - default_rate
	# 重要订单接收频次
	important_freq = np.zeros(402)
	for i in range(data_order2.shape[1]):
		index = np.argsort(data_order2[:, i])[::-1][0:20]
		important_freq[index] += 1
	# 供应商细分市场份额
	all = np.zeros(402)
	for i in range(len(all)):
		if data_supplier[i, 1] == 'A':
			all[i] = np.sum(data_supply2[data_supplier[:, 1] == "A", :])
		elif data_supplier[i, 1] == 'B':
			all[i] = np.sum(data_supply2[data_supplier[:, 1] == "B", :])
		else:
			all[i] = np.sum(data_supply2[data_supplier[:, 1] == "C", :])
	segmentation_market_share = data_supply2.sum(axis=1) / all
	
	# 整合各特征
	data_feature = np.vstack([supply_num, supply_stability2, supply_shift2, order_num, order_stability2, important_freq,
							  segmentation_market_share, vacancy_rate2, default_rate2]).T
	
	# 特征数据输出
	data_out = pd.DataFrame(data_feature)
	data_out.columns = ['供货数量', '供货稳定指数', '供货偏移指数', '订货数量', '订货稳定指数', '重要订单接受频次',
						'供应商细分市场份额', '供应商占用率', '供应商守约率']
	data_out.to_excel('第1问参数数值.xlsx')
	
	# 特征归一化处理
	from sklearn.preprocessing import MinMaxScaler
	
	feature_scaled = deepcopy(data_feature)
	feature_scaled[:, [0, 3, 5]] = MinMaxScaler().fit_transform(feature_scaled[:, [0, 3, 5]])
	
	# 主成分加权
	from sklearn.decomposition import PCA
	
	pca = PCA(svd_solver='full')
	feature_pca = pca.fit_transform(feature_scaled)
	variance = pca.explained_variance_ratio_.reshape(-1, 1)
	supplier_score = feature_pca @ variance
	components = pca.components_
	pd.DataFrame(supplier_score).to_excel('供应商得分.xlsx')  # 输出402家企业加权得分
	pd.DataFrame(np.hstack((variance, components))).to_excel('主成分.xlsx')  # 输出主成分的参数
	
	# 计算各企业最大供应量
	max_offer = data_supply2.max(axis=1).reshape(-1, 1)
	a = np.hstack((supplier_score, max_offer))
	a = a[np.argsort(a[:, 0])[::-1]]  # a从大到小排序
	m = np.array([a[:i, 1].sum() for i in range(50)])
	

		
\end{lstlisting}

\end{document}
